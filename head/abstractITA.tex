\chapter*{Italian abstract}
Quantum generative models leverage the probabilistic structure of quantum mechanics to learn and reproduce complex data distributions beyond classical capabilities. Among these, Born machines encode probabilities via quantum amplitudes, offering efficient sampling and expressive power. However, their scalability is hindered by issues such as barren plateaus. In this thesis, we investigate Tensor Network Born Machines (TNBMs), which integrate tensor networks into the Born machine framework to overcome training obstacles while retaining expressive power. We examine their theoretical underpinnings and practical implementation for learning quantum data distributions. Additionally, we explore the use of positive operator-valued measures (POVMs) for informationally complete measurements, inspired by classical shadow tomography, as a means to enhance the learning capabilities of TNBMs by capturing higher-order correlations and enabling more efficient training.
